%----------------------------------------------------------------------------------------
%	PACKAGES AND OTHER DOCUMENT CONFIGURATIONS
%----------------------------------------------------------------------------------------

\documentclass[12pt]{article}
\usepackage[english]{babel}
\usepackage[utf8x]{inputenc}
\usepackage{amsmath}
\usepackage{graphicx}
\usepackage[colorinlistoftodos]{todonotes}
%\usepackage{fullpage}

\begin{document}

\begin{titlepage}

\newcommand{\HRule}{\rule{\linewidth}{0.5mm}} % Defines a new command for the horizontal lines, change thickness here

\center % Center everything on the page

%----------------------------------------------------------------------------------------
%	HEADING SECTIONS
%----------------------------------------------------------------------------------------

\textsc{\LARGE University of Amsterdam}\\[0.5cm] % Name of your university/college
\textsc{\Large System and Network Engineering}\\[0.5cm] % Major heading such as course name
\textsc{\large Security of Systems and Networks}\\[1cm] % Minor heading such as course title


%----------------------------------------------------------------------------------------
%	LOGO SECTION
%----------------------------------------------------------------------------------------

\includegraphics[scale=0.1]{images/uva-logo.png}\\[1cm] % Include a department/university logo - this will require the graphicx package
 
%----------------------------------------------------------------------------------------


%----------------------------------------------------------------------------------------
%	TITLE SECTION
%----------------------------------------------------------------------------------------

\HRule \\[0.4cm]
{ \huge \bfseries Exhaustive Search on URL Shorteners}\\[0.4cm] % Title of your document
\HRule \\[1cm]
 
%----------------------------------------------------------------------------------------
%	AUTHOR SECTION
%----------------------------------------------------------------------------------------


\begin{minipage}{0.4 \textwidth}
\begin{flushleft} \large
Alexandros Stavroulakis\\
\emph{Alexandros.Stavroulakis@os3.nl}\\[0.5cm]
\end{flushleft}
\end{minipage}
\hfill
\begin{minipage}{0.4 \textwidth}
\begin{flushright} \large
Xavier Torrent Gorj\'{o}n\\
\emph{Xavier.TorrentGorjon@os3.nl}\\[0.5cm]
\end{flushright}
\end{minipage}\\[1cm]

\begin{minipage}{0.5 \textwidth}
\begin{center} \large
Nikolaos Petros Triantafyllidis\\
\emph{Nikolaos.Triantafyllidis@os3.nl}\\[0.5cm]
\end{center}
\end{minipage}\\[3cm]

{\large \today} % Date, change the \today to a set date if you want to be precise

\end{titlepage}

\begin{abstract}
Our names are Alex, Nick and Xavi and we rule. Alex is the the ruthless ruler. Nick is the handsome hunk. Xavi is the cute little being that we use as a teddy bear. Our names are Alex, Nick and Xavi and we rule. Alex is the the ruthless ruler. Nick is the handsome hunk. Xavi is the cute little being that we use as a teddy bear. Our names are Alex, Nick and Xavi and we rule. Alex is the the ruthless ruler. Nick is the handsome hunk. Xavi is the cute little being that we use as a teddy bear.
\end{abstract}
\newpage

%
\section{Introduction}
%
This project is gonna change the history of systems security. 
%
% If you have a question, please use the support box in the bottom right of the screen to get in touch.
%
% \section{Some \LaTeX{} Examples}
% \label{sec:examples}
%
% \subsection{Sections}
%
% Use section and subsection commands to organize your document. \LaTeX{} handles all the formatting and numbering automatically. Use ref and label commands for cross-references.
%
% \subsection{Comments}
%
% Comments can be added to the margins of the document using the \todo{Here's a comment in the margin!} todo command, as shown in the example on the right. You can also add inline comments too:
%
% \todo[inline, color=green!40]{This is an inline comment.}
%
% \subsection{Tables and Figures}
%
% Use the table and tabular commands for basic tables --- see Table~\ref{tab:widgets}, for example. You can upload a figure (JPEG, PNG or PDF) using the files menu. To include it in your document, use the includegraphics command as in the code for Figure~\ref{fig:frog} below.
%
% % Commands to include a figure:
% \begin{figure}
% \centering
% %\includegraphics[width=0.5\textwidth]{frog.jpg}
% \caption{\label{fig:frog}This is a figure caption.}
% \end{figure}
%
% \begin{table}
% \centering
% \begin{tabular}{l|r}
% Item & Quantity \\\hline
% Widgets & 42 \\
% Gadgets & 13
% \end{tabular}
% \caption{\label{tab:widgets}An example table.}
% \end{table}
%
% \subsection{Mathematics}
%
% \LaTeX{} is great at typesetting mathematics. Let $X_1, X_2, \ldots, X_n$ be a sequence of independent and identically distributed random variables with $\text{E}[X_i] = \mu$ and $\text{Var}[X_i] = \sigma^2 < \infty$, and let
% $$S_n = \frac{X_1 + X_2 + \cdots + X_n}{n}
%       = \frac{1}{n}\sum_{i}^{n} X_i$$
% denote their mean. Then as $n$ approaches infinity, the random variables $\sqrt{n}(S_n - \mu)$ converge in distribution to a normal $\mathcal{N}(0, \sigma^2)$.
%
% \subsection{Lists}
%
% You can make lists with automatic numbering \dots
%
% \begin{enumerate}
% \item Like this,
% \item and like this.
% \end{enumerate}
% \dots or bullet points \dots
% \begin{itemize}
% \item Like this,
% \item and like this.
% \end{itemize}
%
% We hope you find write\LaTeX\ useful, and please let us know if you have any feedback using the help menu above.

\end{document}