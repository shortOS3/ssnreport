%----------------------------------------------------------------------------------------
%	PACKAGES AND OTHER DOCUMENT CONFIGURATIONS
%----------------------------------------------------------------------------------------

\documentclass[12pt]{article}
\usepackage[english]{babel}
\usepackage[utf8x]{inputenc}
\usepackage{amsmath}
\usepackage{graphicx}
\usepackage[colorinlistoftodos]{todonotes}
\usepackage{ragged2e}
\usepackage[none]{hyphenat}
%\usepackage{fullpage}

\begin{document}

\begin{titlepage}

\newcommand{\HRule}{\rule{\linewidth}{0.5mm}} % Defines a new command for the horizontal lines, change thickness here

\center % Center everything on the page

%----------------------------------------------------------------------------------------
%	HEADING SECTIONS
%----------------------------------------------------------------------------------------

\textsc{\LARGE University of Amsterdam}\\[0.5cm] % Name of your university/college
\textsc{\Large System and Network Engineering}\\[0.5cm] % Major heading such as course name
\textsc{\large Security of Systems and Networks}\\[1cm] % Minor heading such as course title


%----------------------------------------------------------------------------------------
%	LOGO SECTION
%----------------------------------------------------------------------------------------

\includegraphics[scale=0.1]{images/uva-logo.png}\\[1cm] % Include a department/university logo - this will require the graphicx package
 
%----------------------------------------------------------------------------------------


%----------------------------------------------------------------------------------------
%	TITLE SECTION
%----------------------------------------------------------------------------------------

\HRule \\[0.4cm]
{ \huge \bfseries Exhaustive Search on URL Shorteners}\\[0.4cm] % Title of your document
\HRule \\[1cm]
 
%----------------------------------------------------------------------------------------
%	AUTHOR SECTION
%----------------------------------------------------------------------------------------


\begin{minipage}{0.4 \textwidth}
\begin{flushleft} \large
Alexandros Stavroulakis\\
\emph{Alexandros.Stavroulakis@os3.nl}\\[0.5cm]
\end{flushleft}
\end{minipage}
\hfill
\begin{minipage}{0.4 \textwidth}
\begin{flushright} \large
Xavier Torrent Gorj\'{o}n\\
\emph{Xavier.TorrentGorjon@os3.nl}\\[0.5cm]
\end{flushright}
\end{minipage}\\[1cm]

\begin{minipage}{0.5 \textwidth}
\begin{center} \large
Nikolaos Petros Triantafyllidis\\
\emph{Nikolaos.Triantafyllidis@os3.nl}\\[0.5cm]
\end{center}
\end{minipage}\\[3cm]

{\large \today} % Date, change the \today to a set date if you want to be precise

\end{titlepage}

\tableofcontents
\newpage

\begin{abstract}
%\justify
\noindent
NOTE TO TEAM: This is just a first attempt on an abstract that can work as a guiding light. We'd better write the abstract after the report is finished. Which makes more sense. Peace. And love. \\[0.5cm]
In this project we focus on URL shortening services, from a security point of view.\\ Our first aim is to determine the feasibility of an exhaustive mapping of all the short links to their respective long urls, estimating the cost in both time and computational resources. Secondly we try to discover the nature and the amount of sensitive (usernames, passwords, system configurations, user details, etc.) data that has been deposited to such services, and eventually pinpoint security holes that might have been leaked through them. Our final aim is to try and determine if there is some sort of mapping relationship between the long and short urls. \\
The research methodologies and software tools used for the project are described in detail. The results and interesting findings are presented and the appropriate discretion is applied where deemed necessary. 


\end{abstract}
\newpage

%
\section{Introduction}
%
URL shortening refers to the technique of taking any HTTP Uniform Resource Locator (URL) and producing a shortened version that links to the same Web resource, by issuing an HTTP redirect response. The purpose of this technique is to transform large (sometimes hundreds of characters long) and very descriptive URLs to something that is much sorter, easier to remember and be shared in an environment where typing space is limited (social media, mobile devices, instant messengers, etc.)\\
This technique has been around since the early 2000s but became really popular by the coming of Twitter, a social medium that only allowed a certain number of characters to be typed in each post (Tweet) of the user, and which started automatically shortening URLs more than 26 characters long. The first website to provide shortening services was tinyurl.com, with other similar services including, among others, wp.me (by Wordpress), goo.gl (by Google) and bit.ly, with the last two being the most popular. \\
This report focuses on certain security issues that arise by the use of such services. The rest of this chapter is dedicated to the description of the problem we will be examining, presentation of previous work on this domain and mention of certain ethical implications that arise from our study. The second chapter is a description of the URL shortening methods in general and the two services that have been used in this study (goo.gl and bit.ly) in particular. The third chapter presents the research methods and software tools that we have designed, developed and used in this project. The fourth chapter demonstrates the results that have been produced by our research and the security implications that arise in terms of user privacy and system security. The next chapter is a discussion about suggestions and solutions that could help mediate the security problems of such services. The last chapter summarises the conclusions of the project and proposes ways to improve the current work. 

\subsection{Problem Description}



%
% If you have a question, please use the support box in the bottom right of the screen to get in touch.
%
% \section{Some \LaTeX{} Examples}
% \label{sec:examples}
%
% \subsection{Sections}
%
% Use section and subsection commands to organize your document. \LaTeX{} handles all the formatting and numbering automatically. Use ref and label commands for cross-references.
%
% \subsection{Comments}
%
% Comments can be added to the margins of the document using the \todo{Here's a comment in the margin!} todo command, as shown in the example on the right. You can also add inline comments too:
%
% \todo[inline, color=green!40]{This is an inline comment.}
%
% \subsection{Tables and Figures}
%
% Use the table and tabular commands for basic tables --- see Table~\ref{tab:widgets}, for example. You can upload a figure (JPEG, PNG or PDF) using the files menu. To include it in your document, use the includegraphics command as in the code for Figure~\ref{fig:frog} below.
%
% % Commands to include a figure:
% \begin{figure}
% \centering
% %\includegraphics[width=0.5\textwidth]{frog.jpg}
% \caption{\label{fig:frog}This is a figure caption.}
% \end{figure}
%
% \begin{table}
% \centering
% \begin{tabular}{l|r}
% Item & Quantity \\\hline
% Widgets & 42 \\
% Gadgets & 13
% \end{tabular}
% \caption{\label{tab:widgets}An example table.}
% \end{table}
%
% \subsection{Mathematics}
%
% \LaTeX{} is great at typesetting mathematics. Let $X_1, X_2, \ldots, X_n$ be a sequence of independent and identically distributed random variables with $\text{E}[X_i] = \mu$ and $\text{Var}[X_i] = \sigma^2 < \infty$, and let
% $$S_n = \frac{X_1 + X_2 + \cdots + X_n}{n}
%       = \frac{1}{n}\sum_{i}^{n} X_i$$
% denote their mean. Then as $n$ approaches infinity, the random variables $\sqrt{n}(S_n - \mu)$ converge in distribution to a normal $\mathcal{N}(0, \sigma^2)$.
%
% \subsection{Lists}
%
% You can make lists with automatic numbering \dots
%
% \begin{enumerate}
% \item Like this,
% \item and like this.
% \end{enumerate}
% \dots or bullet points \dots
% \begin{itemize}
% \item Like this,
% \item and like this.
% \end{itemize}
%
% We hope you find write\LaTeX\ useful, and please let us know if you have any feedback using the help menu above.

\subsection{Previous Work}

\paragraph{}
 Most of the previous work which has been done on URL shortening services is based on the use of short URLs and its correlation with SPAM and phishing techniques, whether that is to prevent spamming or to explain how these two are combined. One example of the latter is how the original URL is masked in a way that the receiver of such a malicious email will not be able to realize the fact that by clicking such a link, he or she will not be redirected to a legitimate website.
 
\paragraph{}
 Another example was the investigation of specific countermeasures take from these particular services to defend against the manipulation of the shortened URLs for malicious purposes; also trying to determine and statistically analyze the extent of spamming given certain geographical locations in which the services were used.

\paragraph{}
 As for the analysis of the URLs as independant links and their respective data, the primary focus was on short URLs collected by popular social media such as Twitter. And the statistics revolved around their popularity lifetimes and the expectancy of the amount of clicks these URLs would get.

\subsection{Ethical Implications}

\section{Shortening Services}

\subsection{Goo.gl}

\subsection{Bit.ly}
	
\section{Research Methodologies}

\subsection{URL Crawling}

\subsubsection{Goo.gl}

\subsubsection{Bit.ly}

\subsection{Data Mining}

\subsubsection{RegEx}

\subsubsection{MongoDB Queries}

\section{Results}

\subsection{What can an attacker do?}

\subsection{What have we found?}

\subsection{User Privacy Implications}

\subsection{Sysem Security}

\subsection{Stats}

\section{Suggestions}

 After studying our research results and having witnessed what kind of options and possible offensive routes there are available for an attacker to choose from, we have some suggestions

\subsection{Awareness}

\subsection{Removal of confidential Information}

\subsection{Automated Warnings}

\section{Conclusions}

\section{Future Work}

\section{References}

\section{Appendix}

\subsection{Personal contribution}

\subsection{Codes}

\end{document}